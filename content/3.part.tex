\section{Proposed solution designs}

The specifics needs of people with ASD regarding urban transportation is increasingly being taken into account in current society and thus new technology development. To answers those needs, new solutions are emerging. These solutions can be divided into three groups, guiding assistance, urban planning, and wearable devices.

\subsection{Guiding systems}

As indicated in articles (1), (3) and (2)! !!!!!, the aim of this solution is to guide the user from point A to point B with a range of parameters regarding the user's environment, mainly trough a phone app. 


The solution developed in articles (1) and (2)! !!!!!, consists of guiding people with ASD towards transport centres and buildings. The solution will be implemented by an application on the user's phone which will suggest a route with fewer people, less noise and other criteria. The idea is to be able to locate people with ASD in the area and guide them by giving them information on their phone.
To locate the person, two approaches are considered. The first approach consists of using a semantic 3D model of the buildings and combines it with 2D-3D recordings of images to provide a 3D localisation of the user. The second approach involves using sensors to locate a beacon generated by the user's phone. To achieve this, the article (1)! !!!! suggests to use AI, which will use information based on surveillance cameras around the user. This AI could be trained according to specific criteria, thus enabling people with ASD to have an itinerary adapted to their specific needs.



The other solution mentioned in article (3)! !!!!!!, involves guiding people with ASD around the city. This solution is also an app that the person can have on their phone. This application would be similar to other guiding apps, but could take into account parameters such as gradient, length, accessibility and time. Using these information, it would be possible to create different routes for each type of user. This would make people with ASD able to plan their routes accordingly to their needs.

\subsection{Urban Planning}

Urban planning gives the city a possibility to build its spaces to accommodate its people This type of arrangement is permanent, and affects the population as a whole rather than only targeting people with ASD (unlike the solutions discussed before). Also unlike the solutions discussed before, no user action is required to benefit from urban planning.

The aim of the urban design presented in this article is to make it easier for people with ASD to move around independently. To achieve this, it is necessary to take into account the various problems that arise in the street and that the modification to the layout must address. These include noise and regularly dangerous situations. To achieve this, the document proposes various solutions that can be used altogether. The first step is to define a route from a car park to a main point in the city. After that, one of the next steps is to reduce car traffic, by creating low-speed islands to discourage through-traffic, adding parking areas at the entrances to neighbourhoods, and reducing the size of roads in favour of pavements. Another step is to mark dangerous areas to make them easier to identify. This can be done by marking the regular route taken with a vertical line (on the ground). In safe areas, the line is blue; in dangerous areas, the line is red, indicating that care should be taken. The final stage presented involves using this route to create parks and quiet spaces, adding trees and benches to the city.

The document (4)! !!!!!! also states that this type of development also benefits people without ASD, by improving the quality of life in the city.

\subsection{Wearable assistive technologies}

This type of technology is to be used by people with ASD as a wearable device. The goal of those devices is to improve the quality of social interaction for people with ASD by suggesting adjustments to their behaviour.

As presented in article (5) ! !!!!!!!, the main idea is to correct the stereotyped behaviour of people with ASD. There are many types of devices with this goal in mind that cover a number of stereotyped behaviours. Here are some of the technologies presented in the article:

Eye contact and Joint Attention : They are another key aspect of social skills training, especially for autistic infants and toddlers. The idea is to used a mobile eye-tracking glasses mounted with a camera to automatically detect patterns of mutual eye-contact and advert the person with ASD if needed.

Proximity and distance : Autistic people have troubles responding appropriately to the physical distance between individuals during social interactions. This technology will evaluate the distance between the user and her/his interaction partner, and indicates whether or not it is a comfortable social space. 

 Atypical Prosody : The acoustic quality of an autistic person’s voice and prosody can be unusual and result in misinterpretation of a form of nonverbal communication (flat tone, loud voice). A system built on Google glasses called SayWAT will constantly monitors the voice volume and pitch of users and provides alerts when atypicality is detected. They  will receive trigger alerts via plain text or animation with different colors in function of the severity of the situation.
 
Stereotypical Behaviors : This idea uses wireless Bluetooth accelerometers worn on the right wrist, back of the waist, and left ankle of a person in conjunction that recognize 7 types of stereotypical behaviors. 

General Social Skills : Most existing WATs target one aspect of social interaction. The exception is MOSOCO (Mobile Social Compass), which targets a range of social skills and is built on the social compass curriculum. It includes 24 core lessons from basic non-verbal communication to more complex social problem solving. It is a mobile application that augments a real life social situation and provides visual and verbal support.
