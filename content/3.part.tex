\section{Proposed solution designs}

Nowadays, the specific needs of people with ASD regarding urban transportation are increasingly being taken into account in the development of new technologies. To answer those needs, new solutions are emerging. These solutions can be divided into three groups, guiding assistance, urban planning and wearable devices.

\subsection{Guiding systems}

The aim of this solution is to guide the user from a point A to a point B with a range of parameters regarding the user's environment, mainly through a phone app\cite{2017BuildingSmartAccessible}\cite{2023AccessibilityStrategiesPromote}. 

The solution consists of guiding people with ASD towards transport centers and buildings\cite{2017BuildingSmartAccessible}. The solution is implemented by an application on the user's phone which suggests a route respecting some criteria defined by the user. For instance, the user can select a route with fewer people or less noise. The idea is to locate autistic individuals in the area and guide them by giving them information on their phone. To locate the person, two approaches are considered. The first approach consists of using a semantic 3D model of the buildings and combining it with 2D-3D recordings of images to provide a 3D localization of the user. The second approach involves using sensors to locate a beacon generated by the user's phone. To achieve this, some researchers suggest using AI, which will use information based on surveillance cameras around the user\cite{2017BuildingSmartAccessible}. This AI could be trained according to specific criteria, thus enabling people with ASD to have an itinerary adapted to their specific needs.

The other solution involves guiding people with ASD around the city\cite{2023AccessibilityStrategiesPromote}. This solution is also an app that the person can have on their phone. This application is similar to other guiding apps, but can take into account parameters such as gradient, length, accessibility and time. Using this information, it is possible to create different routes for each type of user. This would make people with ASD able to plan their routes accordingly to their needs.

\subsection{Urban Planning}

Urban planning gives the city a possibility to build its spaces to accommodate the citizens. This type of arrangement is permanent, and benefits the population as a whole rather than only targeting people with ASD, unlike solutions discussed before. No user action is required to benefit from urban planning.

The aim of this urban design is to make it easier for people with ASD to move around independently\cite{2018MobilityPoliciesExtraSmall}. To achieve this, it is necessary to take into account the various problems that arise on the street and that the modifications to the layout must address. These include noise and regularly dangerous situations. To achieve this, the design proposes various solutions that can be used altogether. The first step is to define a route from a car park to a main point in the city. After that, one of the next steps is to reduce car traffic. It can be achieved by creating low-speed islands to discourage through-traffic, adding parking areas at the entrances of the neighborhoods, or reducing the size of roads in favor of pavements. Another step is to mark dangerous areas to make them easier to identify. This can be done by marking the regular route taken with a vertical line on the ground. In safe areas, the line is blue; in dangerous areas, the line is red, indicating that one should proceed with caution. The final stage presented involves using this route to create parks and quiet spaces, adding trees and benches to the city.

This type of development also benefits people without ASD, by improving the quality of life in the city\cite{2018MobilityPoliciesExtraSmall}.

\subsection{Wearable assistive technologies}

This type of technology is designed for autistic individuals as a wearable device. The goal of those devices is to improve the quality of social interaction for people with ASD by suggesting adjustments to their behavior.

The main idea is to correct the stereotyped behavior of individuals with ASD\cite{2018WearableAssistiveTechnologies}. There are many types of devices with this goal in mind that cover a number of stereotyped behaviors. Here are some of the technologies:

Eye contact and Joint Attention: They are another key aspect of social skills training, especially for autistic infants and toddlers. The idea is to use mobile eye-tracking glasses mounted with a camera to automatically detect patterns of mutual eye-contact and notify the person with ASD if needed.

Proximity and distance: Autistic people have troubles knowing the right physical distance between individuals during social interactions. This technology evaluates the distance between the user and their interaction partner, and indicates whether or not it is a comfortable social space. 

Atypical Prosody: The acoustic quality of an autistic person’s voice and prosody can be unusual and can lead to a misinterpretation of a form of nonverbal communication (flat tone, loud voice). A system built on Google glasses called SayWAT constantly monitors the voice volume and pitch of users and provides alerts when atypicality is detected. They receive trigger alerts via plain text or animation with different colors depending on the severity of the situation.
 
Stereotypical Behaviors: This idea uses wireless Bluetooth accelerometers worn on the right wrist, back of the waist, and left ankle of a person in conjunction that recognize 7 types of stereotypical behaviors. 

General Social Skills: Most existing WATs target one aspect of social interaction. The exception is MOSOCO (Mobile Social Compass), which targets a range of social skills and is built on the social compass curriculum. It includes 24 core lessons from basic non-verbal communication to more complex social problem-solving. It is a mobile application that augments a real life social situation and provides visual and verbal support.
