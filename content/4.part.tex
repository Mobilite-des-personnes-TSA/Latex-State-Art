\section{The limits of the solutions}

As we mentioned previously, numerous solutions exist in order to simplify urban transportation for people with ASD. However, nowadays, we do not yet have a perfect solution. Even though the existing solutions can help people with ASD, they all have multiple flaws. First of all, the resources are limited and not adapted to every person. Moreover, existing infrastructures are sometimes incompatible with newly found solutions, making it difficult to implement aforementioned solutions.

Furthermore, most solutions are not ideal because they force autistics to adapt to neurotypical norms, instead of considering their real needs. We should instead turn our focus towards diagnosing autism better and earlier.

\subsection{The existing solutions are limited}
Despite great efforts to find suitable solutions, most of the existence ones are very far from being sufficient. A lot of autistic people can have trouble driving and most of them prefer to use public transportation \cite{2015ViewpointsAdultsAutism} \cite{2020DevelopingCommunityMobility}, but the public transportation is still lagging behind with regards to certain issues. Indeed, most of the people with ASD feel anxious due to the stress of doing bad things, like miss their stop. Also, the noise, smell, light and motion sickness can be a trouble for them. But most of them still have confidence in their ability to use public transportation \cite{2020ExperiencesYoungAutistic}.

Another problem  is the lack of accessibility of public transportation. This alternative is crucial for the mobility of people with ASD but it exists numerous inequalities regarding its accessibility. For instance, suburbans and rural areas does not have as many public transportation as the center of a big city \cite{2015DetourRightPlace}.

Moreover, the existing mobile apps to help guiding people with ASD in their city are not perfect and can be improved. Some of these apps are only available in one specific city, and their functionalities could be improved in order to be really helpful. For example, the application Viana+Acessível that we presented before (part 3.1) is only available in the city of Viana de Castelo, so people living in other cities cannot benefit of it. Also, this app does not have the possibility to choose a path made of multiple destinations, it does not use crowdsourced data and it has no mention of public transports. In short, this app can be really helpful to people with ASD, but its accessibility is limited to only a few people, and it is a shame that it does not include public transportation, knowing that they are crucial for the mobility of people with ASD \cite{2023AccessibilityStrategiesPromote}.

\subsection{Some of the solutions could not be compatible with the already existing infrastructures}
Some of the solutions to help people with ASD to walk arround the cities consist of reorganising them in order to have a better quality of life, like in one district of the city of Sassari, in Italy, which name is Sardinia \cite{2018MobilityPoliciesExtraSmall}.

Unfortunately, not all the cities in the world will agree to do all these reorganisations. The experimentation in Sardinia was made in a small district, but it would be difficult and expensive to apply this solution to the whole city, let alone a bigger one.

Moreover, in a city where most of the people drive by car and the road is already built, it may be not possible to reduce the size of roads in favor of sidewalks, because it would cost too much money and most of people would be against it. Besides, in some cities, it can also be complicated to build more quiet zones with trees, as they already have too much buildings and roads and there is no more space to build new parks. Building a quiet zone would imply destroying another infrastructure, whose owner would be against. 

These solutions are interesting for future projects but cannot always be applied easily to all infrastructures.

\subsection{Most of the solutions force people with ASD to adapt}
A lot of technologies designed to help people with ASD are criticized because they force autistic people to adapt to the social norms we know instead of really helping them. For example, as mentioned in the paragraph 4.3, this article talks about numerous gadgets which have one thing in common : they see autism as a disability and help autistic people adapt to the social norms \cite{2017UTravelSmartMobility}.However, some autistic people prefer not to see it as a disability, but as a different way of behaving. Therefore, it would be better to have technologies to help them find their own ways of communicating with their partners. 

\subsection{The importance of the diagnosis}
Another type of solution would be to improve the detection of ASD on the children. Like it has been described in the introduction, it is very important to detect this trouble the earlier that is possible. In this way we can cite a solution which uses new technologies like VR. The idea is to make some games that will test the children, this includes tutorial levels, in-game day levels, mental stressor mini-games, and physical exercise mini-games  \cite{2021DesigningSmartVirtual}.

This type of solution unlike the other will not directly improve the usage of transport for the person with ASD. But this can still be a great way of improvement that can be developed. 
