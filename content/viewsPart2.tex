\section{Perspectives and Challenges Faced by Individuals with Autism Spectrum Disorder (ASD) in Transportation}

\subsection{Challenges Encountered}

    Transportation poses a profound array of challenges for individuals with Autism Spectrum Disorder (ASD), as illuminated by in-depth research studies \cite{falkmer_viewpoints_2015} \cite{deka_co-principal_nodate} \cite{haas_experiences_nodate}. 
    
    Among the multiple difficulties faced by this population, an Australian study realised in Sydney  \cite{haas_experiences_nodate} demonstrate that anxiety and stress are the most common challenge faced by young adults with ASD on their daily experiences with transportation. This anxiety is rooted in the constant fear of making errors during travel, like missing crucial stops, and navigating the uncertainty of potential modifications to their transportation routines. This heightened emotional state adds layers of complexity to their routine activity. 
    This study also confirms that sensory challenges emerge as another barrier, encapsulating issues such as balance and motion sickness, sensitivity to environmental noise, tactile sensations, lights, and smells encountered during public transport journeys. These sensory intricacies further intensify the challenges faced by individuals with ASD, making routine travel a daunting prospect. 
    
    A significant concern arises from the heavy dependency on parents and caregivers for transportation, creating a dynamic that introduces inconveniences and stress for both adults with ASD and their immediate support circles \cite{deka_co-principal_nodate}. This dependence not only impacts their autonomy but also has wider implications for the flexibility and convenience of their daily lives.
    The transition period after the conclusion of school, typically around age 21, unveils another layer of challenges, as government support and school transportation cease, thrusting individuals with ASD and their families into a realm of newfound transportation complexities. 
    Limited walking skills among many adults with ASD further contribute to their reliance on others for transportation, impeding their capacity to navigate public spaces independently \cite{deka_co-principal_nodate}. 
    
    Safety concerns associated with driving pose significant obstacles, particularly from the perspective of concerned parents, despite an expressed interest in obtaining a driver's license for the sake of independence \cite{deka_co-principal_nodate}. 

    Moreover, specific concerns about crowding and safety emerge as tangible barriers to public transport for individuals with ASD, while neurotypical adults grapple with obstacles related to service location and timing \cite{falkmer_viewpoints_2015}. 
    
    These multifaceted challenges underscore the urgency for targeted interventions and comprehensive strategies to enhance the transportation experiences of individuals with ASD, ultimately fostering greater independence, accessibility, and inclusion within the broader community.

\subsection{Current Adaptations}

   To mitigate those challenges, individuals with Autism Spectrum Disorder (ASD) employ a range of adaptive strategies to navigate within public transports. 
   
   Leveraging their support circles, including family members and caregivers, proves instrumental in mitigating the stress and anxiety often linked to travel \cite{deka_co-principal_nodate}. 
   
   Planning ahead becomes a cornerstone strategy \cite{haas_experiences_nodate}, allowing these individuals to anticipate potential modifications and familiarize themselves with routes, instilling a sense of control.
   Avoiding crowded spaces, choosing off-peak hours, and developing heightened spatial awareness are additional strategies aimed at minimizing stress during public transport journeys.
   Some individuals find comfort in utilizing bicycles for short-distance travel \cite{falkmer_viewpoints_2015}, fostering a sense of autonomy. The inclination towards self-reliance is evident, as participants express a preference for drawing on personal coping mechanisms rather than seeking external help. Establishing routines and adhering to predictability emerge as effective coping mechanisms, reducing anxiety by introducing familiarity into their travel experiences. 
   
   Technological tools, such as smartphones and applications, play a crucial role in aiding individuals with ASD in planning routes, checking transportation schedules, and enhancing their independence during travel. 
   Managing sensory challenges involves the use of music or headphones to create a controlled auditory environment, providing relief from noise sensitivity \cite{haas_experiences_nodate}. 
   
   Overall, these adaptive strategies underscore the resilience and resourcefulness of individuals with ASD as they navigate the complexities of public transportation.

\subsection{Expectations and Envisaged Solutions}

    Individuals with Autism Spectrum Disorder (ASD) advocate for a comprehensive set of expectations and corresponding solutions to confront the challenges entwined with transportation. 
    
    At the forefront of their aspirations is the collective desire for a reduction in anxiety and stress during travel, underscoring the urgent need for the implementation of sensory-friendly measures \cite{deka_co-principal_nodate}. This includes quieter travel options, dedicated spaces tailored to sensory sensitivities, and clear communication practices to mitigate uncertainty.
    The overarching expectation of improved accessibility fuels a demand for the incorporation of universal design principles into transportation infrastructure, ensuring facilities and services cater to diverse abilities.
    
    The quest for increased independence resonates as a common goal, prompting a call for the development of cutting-edge technologies, user-friendly apps, and robust training programs. These initiatives aim to empower individuals with ASD to navigate public spaces autonomously. 
    
    The prospect of a smooth transition after school elicits a call for integrated travel education within school curricula, specifically embedded in Individualized Education Programs. 
    Addressing the burgeoning interest in driving involves tailoring driver education programs to assuage safety concerns and foster autonomy \cite{deka_co-principal_nodate}. 
    
    Societal expectations revolve around heightened awareness, calling for collaborative awareness campaigns designed to educate the general public on the unique needs of individuals with ASD. Collaboration and integration take center stage, advocating for the establishment of protocols that foster sustainable transportation infrastructure in partnership with Non-Governmental Organizations and dedicated researchers \cite{deka_co-principal_nodate}. 
    The expectation of adopting best practices underscores a global perspective, urging the initiation of global studies to identify and replicate successful strategies. Specialized training for transportation staff becomes a focal point, emphasizing collaboration with the ASD stakeholder community to craft effective training modules. 

    The desire for service expansion, coupled with the creation of a dedicated statewide mobility manager, signifies a collective plea for comprehensive, inclusive, and tailored strategies to address the intricate and diverse needs of individuals with ASD in the realm of transportation.
